\chapter{Introduzione}\label{c:introduzione}

Il machine learning è un sottoinsieme dell'intelligenza artificiale che si occupa di creare sistemi che apprendono e migliorano le loro performance sulla base dei dati che utilizzano. L'idea alla base è che i sistemi possono eseguire dei compiti specifici senza essere programmati per farlo,  imparando dai dati e identificando dei modelli all'interno di essi.

La crescita esponenziale del volume dei dati prodotti ha permesso un grande sviluppo di algoritmi di machine learning,  che al giorno d'oggi hanno raggiunto una diffusione capillare all'interno dei sistemi più utilizzati,  dalle applicazioni web e mobile,  agli assistenti virtuali fino all'assistenza alla guida per le automobili.  In particolare nel campo del marketing,  i siti web di e-commerce utilizzano il machine learning per consigliare agli utenti dei prodotti che potrebbero interessare maggiormante l'acquirente sulla base degli ultimi acquisti o delle ricerche fatte. Nel campo finanziario invece è possibile identificare delle opportunità di investimento studiando i dati di mercato di molte aziende, aiutando così gli investitori a fare investimenti corretti.

Sviluppare applicazioni, che siano desktop, mobile o web app, che utilizzino algoritmi di apprendimento automatico però non è banale, perchè richiede conoscenza di data analysis e di machine learning per poter implementare un algoritmo all'interno della propria applicazione. Esiste però la possibilità di sfruttare algoritmi di machine learning in modo più agevole e veloce per gli sviluppatori, sviluppando le proprie applicazioni in modo che debbano solo interfacciarsi con dei servizi web, i quali gestiranno tutta la parte di machine learning e si occuperanno di fornire le predizioni richieste. Questo è possibile grazie al software Apache PredictionIO che permette di creare dei servizi web con i quali qualsiasi applicazione può interagire tramite richieste HTTP e usare qualsiasi algoritmo di machine learning in modo più facile e veloce rispetto a come sarebbe integrarlo direttamente all'interno dell'applicazione. Grazie a un software come PredictionIO quindi il machine learning diventa accessibile a qualsiasi sviluppatore e non solo ai data scientist.

Ho quindi deciso di sviluppare un'applicazione desktop per la valutazione di aziende in ambito finanziario che permetta agli utenti di avere supporto per gli investimenti. Questa applicazione si interfaccia con il software Apache PredictionIO per avere valutazioni sul futuro andamento dell'azienda e avere consigli su altre aziende simili, sulla base del profilo  e degli attuali interessi di un utente.

Il documento prosegue con il capitolo \ref{c:predictionio}, nel quale viene introdotto e spiegato il funzionamento di Apache PredictionIO, approfondendo la sua architettura e le sue componenti, spiegando infine una possibile via per l'installazione. Nel capitolo \ref{c:applicazione} viene spiegato come è stata sviluppata l'applicazione, le sue funzionalità e la sua struttura, spiegando infine come si integra con PredictionIO. Nel capitolo \ref{c:conclusione} si traggono le conclusioni facendo delle considerazioni finali.