\chapter{Conclusione}\label{c:conclusione}

Lo svolgimento di questo progetto mi ha permesso di approfondire l'argomento del machine learning, non tanto dal punto di vista tecnico ma più dal punto di vista della funzione che esso sta assumendo nel mondo informatico odierno, infatti, nelle sue numerose varianti, è ormai parte integrante di ogni applicazione, che sia web o mobile. Per questo ho deciso di focalizzarmi più sull'integrazione degli algoritmi di machine learning con le applicazioni e su come queste due branche dell'informatica effettivamente interagiscono. Il progetto Apache PredictionIO è molto rappresentativo di questo aspetto, infatti consente di isolare la complessità del machine learning in moduli, chiamati engines, e lasciare allo sviluppatore solamente il compito di modificare questi engines e integrarli con le loro applicazioni.  Progetti simili consentono la diffusione ancora maggiore del machine learning in quanto lo rendono accessibile a tutti gli sviluppatori e non solo ai data scientist.

Lo studio di PredictionIO mi ha permesso di imparare com'è strutturato un progetto software di grandi dimensioni, studiarne l'architettura e come viene implementato a livello pratico. Infatti con l'implementazione degli engines per la classificazione e la raccomandazione ho imparato a modificare codice già scritto, capendone il significato e la struttura per andarlo a modificare adattandolo alle mie esigenze. In secondo luogo ho potuto costruire, sebbene in  locale, dei servizi web, eseguendone il deploy su diverse porte. Infine installando PredictionIO con Docker ho avuto modo di sfruttare e imparare questo software per la containerizzazione molto utilizzato al giorno d'oggi per i vantaggi che offre, quali la leggerezza e la portabilità.

La seconda parte di questo progetto mi ha permesso inoltre di cimentarmi nello sviluppo di un'applicazione, approfondendo il linguaggio Python applicato alla creazione di applicazioni piuttosto che all'analisi dei dati, usando in modo dettagliato le classi e in generale il paradigma di programmazione orientato agli oggetti. Per lo sviluppo dell'applicazione ho avuto anche l'opportunità di studiare e imparare a usare la libreria grafica di Python, TKinter; approfondire l'interazione del linguaggio con un database come MySQL, sfruttando il modulo relativo; e infine la gestione di un'interfaccia per l'interazione con dei servizi web esterni, implementata nella classe Predictor sfruttando le API di PredictionIO disponibili per Python.